\documentclass[notitlepage]{report}

\usepackage{titling}
\usepackage{subcaption}
\usepackage{amsmath}
\usepackage{url}
\usepackage{wrapfig}
\usepackage{amssymb}
\usepackage{breqn}
\usepackage[a4paper, total={6in, 9in}]{geometry}
\newcommand\norm[1]{\left\lVert#1\right\rVert}
\newcommand{\angstrom}{\text{\normalfont\AA}}
\usepackage{natbib}
\usepackage{caption}
\usepackage{graphicx}
\usepackage{siunitx}
\usepackage{hyperref}
\graphicspath{ {./images/} }

\title{Numerical Simulations and Analytical Results Exploring the use of the Generalized Langevin Equation for Simulations of Atomic Scale Surface Dynamics}
\author{Blind Grading Number: 4631L \\[0.3cm]{Supervised by Dr. John Ellis}}
\date{17th of May 2021}

\newcommand{\mev}{\si{\milli\eV}}
\newcommand{\thz}{\si{\tera\hertz}}
\newcommand{\ns}{\si{\nano\second}}
\newcommand{\ps}{\si{\pico\second}}
\newcommand{\ips}{\si{\per\pico\second}}
\newcommand{\fs}{\si{\femto\second}}
\newcolumntype{P}[1]{>{\centering\arraybackslash}p{#1}}
\newcolumntype{P}[1]{>{\centering\arraybackslash}p{#1}}
\newcolumntype{P}[1]{>{\centering\arraybackslash}p{#1}}
\newcolumntype{P}[1]{>{\centering\arraybackslash}p{#1}}
\newcommand{\expnumber}[2]{{#1}\mathrm{e}{#2}}

\begin{document}
\maketitle
\thispagestyle{empty}

\begin{abstract}

	Langevin equations are widely used as a computationally cheap model of a subset of the degrees of freedom in a complex system. The white noise present in the original Langevin equation, while often a good approximation, does not capture the colored noise in most real systems. This necessitates the introduction of the generalized Langevin equation (GLE). Recent results in the field of surface dynamics \cite{Ward} have used the GLE to explain experimental data for the activated diffusion of lithium on copper(111) by introducing band-limited noise. However, the noise spectrum required was band-limited to around $\SI{1.25}{\thz}$, much lower than the $\SI{7.4}{\thz}$ \cite{sinha} which the phonon cutoff frequency of copper would suggest. The following report presents a careful evaluation of the performance of the GLE in the presence of a corrugated background potential through analytic analysis and numerical comparison with full 3D molecular dynamics (MD) simulations. The analytic work provides an interpretation of the parameters in the GLE in the wide bandwidth regime and proposes a simple technique for the extraction of parameters from experimental/MD results. A semi-realistic classical MD simulation of lithium on copper(111) is constructed with the correct copper phonon spectrum. The GLE with a frequency cutoff of $\SI{7.4}{\thz}$ is shown to reproduce the simulated MD results extremely well leading to the proposal that an additional, unknown, noise filter is present in lithium on copper(111) which reduces the effective bandwidth seen by the lithium. Since lithium is an extremely light atom with a thermal de Broglie wavelength on the order of $\SI{0.5}{\angstrom}$ at $\SI{140}{\kelvin}$, this filter may be quantum mechanical in nature.  

\end{abstract}


\vspace{1cm}
\begin{center}
\includegraphics[width=0.80\textwidth]{md_top_down}
\end{center}

\pagebreak

\section*{Acknowledgments}

I would like to thank my supervisor, Dr. John Ellis, for the time, effort and enthusiasm he has invested in this project.
I would like to thank the Skye Foundation, Oppenheimer Memorial Trust and Cambridge Trust for providing the financial means for me to pursue the degree for which this report was submitted.

\addvspace{2cm}

\section*{Statement of Authenticity}

Except where specific reference is made to the work of others, this work is original and has not been already submitted either wholly or in part to satisfy any degree requirement at this or any other university.

\addvspace{2cm}

\section*{List of Acronyms}

\begin{itemize}
  \item GLE - Generalized Langevin equation
  \item MD - Molecular dynamics simulation
  \item HSEM - Helium spin echo microscopy
  \item ISF - Intermediate scattering function
  \item fcc - Face centered cubic crystal
  \item dof - Degrees of freedom
\end{itemize}

\tableofcontents
\listoffigures
\listoftables

\include{introduction}
\include{analytic_gle}
\include{simulation_setup}
\include{results}

\chapter{Conclusions and Outlook}

A careful analysis of the thermalization properties of the GLE in various background potentials has been performed. The results show that the introduction of a cutoff in the noise spectrum introduces additional variance in the temperature fluctuations of the system. Additionally, the Velocity Verlet integration technique is shown to produce a mean simulation temperature which is $1-2\%$ too high when applied to the GLE in a corrugated potential.

A first-principles derivation of the ISF and energy auto-correlation function of a particle obeying the GLE in a harmonic background potential has been presented. These expressions have been evaluated analytically for the first time in the special case of an exponential memory kernel, $K(t)=\frac{1}{\tau}e^{-\frac{t}{\tau}}$, and used to provide interpretations of the GLE parameters in the small $\tau$ regime. These results serve as a useful tool for extracting GLE parameters from the ISF and energy auto-correlation function of a MD simulation. These parameter extraction techniques have potential to be applied directly to experimental data.

A semi-realistic, classical, 3D molecular dynamics simulation of lithium on copper(111) has been constructed with one to two orders of magnitude more substrate atoms than similar simulations in previous work. The simulation has been fully validated, analyzed and compared with the known properties of lithium on copper(111). In particular, the simulation faithfully reproduces the phonon spectrum of real copper (with a phonon cutoff frequency of $\SI{7.4}{\tera\hertz}$) and therefore fully captures the classical notion of the stochastic noise seen by an adatom. The simulation has been used to show that this classical simulation of lithium on copper(111) fails to reproduce the pre-exponential factor of the hopping rate as well as ratio of fast to slow motion seen in the experimental data.

A corresponding 2D GLE simulation with an exponential memory kernel has been constructed with parameters extracted directly from the MD simulation using the aforementioned parameter extraction technique. This simulation has been shown to produce ISFs which match the MD simulations extremely well. Together with results from \cite{Ward} which show that the experimental lithium on copper data is explained using a GLE with a frequency cutoff of around $\SI{1.25}{\tera\hertz}$, this leads to the conclusion that the effective noise seen by lithium on copper(111) is far more band limited than a classical model would predict. This suggests an additional noise filter of unknown origin is present in the system which effectively band limits the noise spectrum seen by the lithium to around $\SI{1.25}{\tera\hertz}$. Since lithium is a very light atom with a thermal de Broglie wavelength on the order of $\SI{0.5}{\angstrom}$ (smaller than but still comparable to the copper lattice spacing of $\SI{2.54}{\angstrom}$), this filter may be quantum mechanical in nature.

Using a corrugated potential extracted from the full MD simulation, the effect of the parameters $\eta$ and $\tau$ on the ISFs produced in GLE simulations are derived numerically. The simulation results re-affirm the expectation that a frequency cutoff much higher than the typical hopping rate observed in the MD simulation of around $\SI{0.93}{\ips}$(corresponding to very small $\tau$) has very little effect on the long term hopping rate of the particle. The numerical results suggest an asymptotic relationship of $\alpha \propto \sqrt{\frac{\eta}{\tau}}$ for the ISF dephasing rate at larger $\tau$ which is not fully understood and may provide an avenue for further research within transition state theory. 

\appendix

\chapter{Derivation of Analytic Expressions for the GLE in a Harmonic Well} \label{apx:analytic_gle_appendix}

This appendix provides a first-principles derivation of analytic expressions for the ISF and energy auto-correlation function of the GLE within a background harmonic potential well. Some of these expressions have been independently derived in an alternative form in \cite{Townsend_2018}, however some of these expressions are evaluated for the first time. 

\section{The Greens Function of the GLE in a Harmonic Potential}

The generalized Langevin equation is given by

$$
m\ddot{x} + m\eta \int_{-\infty}^{\infty} dt' K(t - t')v(t') + m\omega_0^2 x = \int_{-\infty}^{\infty} dt' K(t - t')f(t')
$$
\\
for an arbitrary memory kernel $K(t)$ in a harmonic potential background with natural frequency $\omega_0$. The random force is given by $f(t)$ and has variance $\left<f_i(t)f_j(t')\right>=\delta_{ij}\delta(t-t')\sigma^2 = \delta_{ij}\delta(t-t')2kTm\eta$ where $m$ is the particle mass and $\eta$ is a friction constant in units of inverse time. If required, causality is assumed to be implicit to $K(t)$. Fourier transforming both sides and using the convolution theorem,

$$
-m\omega^2 \tilde{x}(\omega) + i m \eta \omega \tilde{x}(\omega) \tilde{K}(\omega) + m\omega_0^2 \tilde{x}(\omega) = \tilde{f}(\omega) \tilde{K}(\omega)
$$
$$
\implies \tilde{x} = \frac{1}{m} \frac{\tilde{f} \tilde{K}}{-\omega^2 + i \eta \omega \tilde{K} + \omega_0^2} = \frac{1}{m} \tilde{f} \tilde{F}.
$$
\\
The Green's function of the GLE, $F(t)$, has been introduced so that the trajectory of a particle is given by $x(t) = \frac{1}{m}\int_{-\infty}^{\infty}dt'F(t-t')f(t')$.

\section{The ISF of a Particle Obeying the GLE in a Harmonic Potential Well \label{isf_gle_well}}
\label{ch:gle_derivation}

The intermediate scattering function is given by the spatial Fourier transform of the van Hove pair correlation function $G\left(\vec{r},t\right)$. For a single particle, this is equivalent to $P\left(\vec{r}, t\right)$, the probability of finding the particle at $\vec{r}$ at time $t$ given that it was at the origin at $t=0$ \cite{vanhove}. For a single particle trajectory, $\vec{r}(t)$, this given by 

$$
ISF(\Delta \vec{K}, t) = \int d\vec{R} e^{-i \Delta \vec{K} \cdot \left(\vec{R} - \vec{r}(0)\right)} P(\vec{R}, t) = \int d\vec{R} e^{-i \Delta \vec{K} \cdot \left(\vec{R} - \vec{r}(0)\right)} \delta(\vec{R} - \vec{r}(t)) = e^{-i \Delta \vec{K} \cdot \left(\vec{r}(t) - \vec{r}(0)\right)}
$$
\\
Using the Green's function $F(t)$, $\vec{r}(t) - \vec{r}(0) = \frac{1}{m} \int \frac{d\omega}{2\pi} \left(e^{i\omega t} - 1\right) \tilde{F} \tilde{f}$. Substituting this into the above results in

$$
ISF(\Delta \vec{K}, t) = e^{- \frac{i}{m} \int \frac{d\omega}{2\pi} \left(e^{i\omega t} - 1\right) \tilde{F} \left(\Delta \vec{K} \cdot \tilde{f}\right))}
$$
\\
Expanding out the exponential and taking an ensemble average yields
$$
\left<ISF(\Delta \vec{K}, t)\right> = \sum_{n=0}^{\infty} \left(- \frac{i}{m}\right)^n \frac{1}{n!} \left< \left( \int \frac{d\omega}{2\pi} \left(e^{i\omega t} - 1\right) \tilde{F} \left(\Delta \vec{K} \cdot \tilde{f}\right)\right)^n\right>
$$
\\
\begin{equation}
= \sum_{n=0}^{\infty} \left(- \frac{i}{m}\right)^n \frac{1}{n!} \left( \int \frac{d\omega}{2\pi} \left(e^{i\omega t} - 1\right) \tilde{F}\right)^n \left< \left(\Delta \vec{K} \cdot \tilde{f}\right)^n\right> \label{eq:isf_1}
\end{equation}
\\
I have abused notation in the last line to summarize $n$ integrals over $\omega_1$ through $\omega_n$. For the remainder of the derivation it is assumed that the random force is both isotropic and has a zero mean. These constraints can be relaxed and this may be used to model materials with an-isotropic noise spectra such as materials with large differences in the phonon frequency cut-offs for the various phonon polarization directions.
Taking a closer look at $\left< \left(\Delta \vec{K} \cdot \tilde{f}\right)^n\right>$ in one dimension,
$$
\left< \left(\Delta \vec{K} \cdot \tilde{f}\right)^n\right> = \left|\Delta \vec{K}\right|^n \left< \tilde{f}\left(\omega_1\right) \ldots \tilde{f}\left(\omega_n\right)\right>.
$$
From the isotropy and zero mean of $f$ it follows that the above vanishes for odd $n$. For even $n$, the expectation is given by summing over the product of all possible pairwise expectations of $\left< \tilde{f}\left(\omega_1\right) \ldots \tilde{f}\left(\omega_n\right) \right>$ \footnote{This follows from the vanishing higher order cumulants of white noise. See David Tong's notes \cite{Tong}.},
$$
\left< \left(\Delta \vec{K} \cdot \tilde{f}\right)^{2n}\right> = \left|\Delta \vec{K}\right|^{2n} \frac{1}{2^nn!} \sum_P \left< \tilde{f}\left(\omega_{P_1}\right) \tilde{f}\left(\omega_{P_2}\right)\right> \ldots \left< \tilde{f}\left(\omega_{P_{2n-1}}\right) \tilde{f}\left(\omega_{P_{2n}}\right)\right>
$$
$$
= \left|\Delta \vec{K}\right|^{2n} \frac{\left(2\pi\sigma^2\right)^n}{2^nn!} \sum_P \delta\left(\omega_{P_1} + \omega_{P_2}\right) \ldots \delta\left(\omega_{P_{2n-1}} + \omega_{P_{2n}}\right)
$$
The last step follows immediately from the Fourier transform of $\left<f(t)f(t')\right>=\delta(t-t')\sigma^2$. Substituting into \ref{eq:isf_1} yields
$$
\sum_{n=0}^{\infty} \left(- \frac{i}{m}\right)^{2n} \frac{1}{(2n)!} \left( \int \frac{d\omega}{2\pi} \left(e^{i\omega t} - 1\right) \tilde{F}\right)^{2n} \left|\Delta \vec{K}\right|^{2n} \frac{\left(2\pi\sigma^2\right)^n}{2^nn!} \sum_P \delta\left(\omega_{P_1} + \omega_{P_2}\right) \ldots \delta\left(\omega_{P_{2n-1}} + \omega_{P_{2n}}\right)
$$
$$
= \sum_{n=0}^{\infty} \left(- \frac{i}{m}\right)^{2n} \frac{1}{(2n)!} \left( \int \frac{d\omega}{2\pi} \left|\left(e^{i\omega t} - 1\right) \tilde{F}\right|^2\right)^{n} \left|\Delta \vec{K}\right|^{2n} \frac{\left(\sigma^2\right)^n(2n)!}{2^nn!}
$$
$$
= \sum_{n=0}^{\infty} \frac{1}{n!} \left(-\frac{|\Delta \vec{K}|^2 \sigma^2}{m^2} \int \frac{d\omega}{2\pi}\left(1 - \cos\left(\omega t\right)\right) \left| \tilde{F} \right|^2\right)^n
$$
\begin{equation}
= \exp\left(-\frac{|\Delta \vec{K}|^2 \sigma^2}{m^2} \int \frac{d\omega}{2\pi}\left(1 - \cos\left(\omega t\right)\right) \left| \tilde{F} \right|^2\right) \label{isf_explicity_norm}
\end{equation}
\\
From equation \ref{isf_explicity_norm} it is clear the ISF is normalized to 1 at $t=0$. Since $F(t)$ is a real function, $|\tilde{F}|^2$ must be even, so $\int \frac{d\omega}{2\pi}\left(\cos\left(\omega t\right)\right) \left| \tilde{F} \right|^2 = \int \frac{d\omega}{2\pi}\ e^{i \omega t} \left| \tilde{F} \right|^2$ which is nothing but the auto-correlation of $F(t)$ (denoted by $C_F(t)$). Moreover, $\int \frac{d\omega}{2\pi}\left| \tilde{F} \right|^2$, by the residue theorem, is nothing more than the sum of the residues of the poles of $\left|\tilde{F}\right|^2$ in the upper (or lower since the function is symmetric) half complex plane. So our final expression for a general isotropic memory kernel is
\begin{equation}
\left<ISF(\Delta \vec{K}, t)\right> = ISF\left(\Delta \vec{K}, \infty \right)\exp\left(\frac{|\Delta \vec{K}|^2 \sigma^2}{m^2} C_F\left(t\right) \right). \label{isf_compact} 
\end{equation}
\\
We are able to define $ISF\left(\Delta \vec{K}, \infty \right)= \exp \left( -\frac{|\Delta \vec{K}|^2 \sigma^2}{m^2} \int \frac{d\omega}{2\pi} \left| \tilde{F} \right|^2 \right)$ since at large times, provided $F(t \rightarrow \infty)$ tends to $0$ fast enough, $C_F(t \rightarrow \infty) = 0$.

\section{The Kinetic Energy Auto-correlation Function \label{kinetic_autocorrelation}}

The velocity of a particle obeying the GLE may be written in terms of the Green's function as $\dot{x}(t) = \frac{i}{m}\int\frac{d\omega}{2\pi} \omega e^{i\omega t}\tilde{f}(\omega)\tilde{F}(\omega)$. The kinetic energy auto-correlation function at equilibrium is therefore given by
$$
\left<E(0)E(t)\right>=\frac{m^2}{4}\left<\dot{x}(0)^2\dot{x}(t)^2\right>=\frac{1}{4m^2}\left(\int\frac{d\omega}{2\pi}\omega\tilde{F}\right)^4 e^{i\left(\omega_3 + \omega_4 \right)t} \left<\left(\tilde{f}(\omega_1)\cdot\tilde{f}(\omega_2)\right)\left(\tilde{f}(\omega_3)\cdot\tilde{f}(\omega_4)\right)\right>
$$
Where I have once again abused notation to summarize four integrals over $\omega_1\cdots\omega_4$. By expanding the dot product component wise and taking care to sum over all the products of pairwise expectations (as in \cite{Tong}), in two dimensions,  $\left<\left(\tilde{f}(\omega_1)\cdot\tilde{f}(\omega_2)\right)\left(\tilde{f}(\omega_3)\cdot\tilde{f}(\omega_4)\right)\right>$ is given by 

$$
2\left(2\pi\sigma^2\right)^2\left(2\delta(\omega_1+\omega_2)\delta(\omega_3+\omega_4) + \delta(\omega_1+\omega_2)\delta(\omega_3+\omega_4) + \delta(\omega_1+\omega_2)\delta(\omega_3+\omega_4)\right).
$$
\\
Therefore in 2 dimensions,
$$
\left<E(0)E(t)\right>=\frac{\sigma^4}{m^2}\left(\left(\int\frac{dw}{2\pi}\omega^2\left|\tilde{F}(\omega)\right|^2\right)^2 + \left(\int\frac{dw}{2\pi}e^{i\omega t}\omega^2\left|\tilde{F}(\omega)\right|^2\right)^2\right).
$$

\section{Simplifications for an Exponential Memory Kernel}

\subsection{Evaluating integrals of the form $\int\frac{dw}{2\pi} g\left(\omega\right) \left|\tilde{F}\left(\omega\right)\right|^2$} 

Given an exponential memory kernel $K(t) = \theta(t) \frac{1}{\tau} e^{-\frac{t}{\tau}}$, integrals of the form $\int\frac{dw}{2\pi} g\left(\omega\right) \left|\tilde{F}\left(\omega\right)\right|^2$ for well behaved functions $g$ may be evaluated using the residue theorem. The Fourier transform of the kernel is given by $\tilde{K}\left(\omega\right) = \frac{1}{1+i\omega\tau}$ which may be easily verified using the residue theorem. Substituting $\tilde{K}$ into the formula for $\tilde{F}$ and simplifying gives
$$
\tilde{F} = \frac{-1}{i\tau\omega^3 + \omega^2 - i\omega(\omega_0^2\tau + \eta) - \omega_0^2} =: \frac{-1}{P(\omega)}
$$
Where the 3rd order polynomial $P(\omega)$ has been introduced. Since $P(i\omega)$ is a polynomial with real co-efficients, it follows from the complex conjugate root theorem that for any root $z$ of $P(\omega)$, $-z^*$ is also a root of $P(\omega)$. This allows us to make the ansatz that $P$ may be factorized as $P(\omega) = i\tau\left(\omega - \chi\right)\left(\omega + \chi^*\right)\left(\omega - i\eta_1\right)$ for some $\chi\in\mathbb{C}$ and ${\eta_1\in\mathbb{R}}$. While it is possible that in certain regions of the $\tau,\eta,\omega_0$ parameter space $P(\omega)$ has three imaginary roots (analogous to the damped harmonic oscillator or \cite{Townsend_2018}), these parameter ranges were not encountered in the course of this project and were not further investigated. Furthermore, if the Greens function $F(t)$ is to be causal, all three roots must occur in the upper half plane \cite{Runkel}. We therefore find that the pole structure of $\left|\tilde{F}\right|^2$, as shown in Figure \ref{fig:pole_structure}. 
\\
\begin{figure}
	\centering
	\includegraphics[width=0.5\textwidth]{pole_structure}
	\caption{\label{fig:pole_structure} The pole structure of the auto-correlation of the GLE Green's function $F(t)$ with an exponential memory kernel.}
\end{figure}

$\int\frac{dw}{2\pi} g\left(\omega\right) \left|\tilde{F}\left(\omega\right)\right|^2$ may now be evaluated using the residue theorem (closing the contour at $i\omega\rightarrow\infty$) provided that $g$ does not grow too fast as $\Im(\omega)\rightarrow\infty$. Assuming $g(\omega)$ has no poles and that $g(-\omega^*)=g(\omega)^*$ (which is useful for the purposes of this report),
\begin{equation}
	\int\frac{dw}{2\pi} g\left(\omega\right) \left|\tilde{F}\left(\omega\right)\right|^2 =  -\frac{1}{2 \tau^{2}}\left(\frac{1}{2 \chi' \chi''} \operatorname{Re}\left(\frac{g(\chi)}{\chi\left(\chi^{2}+\eta_{1}^{2}\right)}\right)+\frac{g(i\eta_{1})}{\left(|\chi|^{2}+n_{1}^{2}\right)^{2} \eta_{1}}\right) \label{eqn:integral_over_CF}
\end{equation}

\subsection{Evaluating the ISF and Energy Auto-correlation Function for an Exponential Memory Kernel}

Using equation \ref{eqn:integral_over_CF} the results derived in Section \ref{isf_gle_well} and \ref{kinetic_autocorrelation} can be evaluated for an exponential memory kernel $K(t)=\frac{1}{\tau}e^{-\frac{t}{\tau}}$. Substituting the expressions for the ISF and kinetic energy auto-correlation into equation \ref{eqn:integral_over_CF} and simplifying results in,
\begin{equation}
	ISF\left(\Delta \vec{K}, t\right) = ISF\left(\Delta \vec{K}, \infty\right) \exp\left(\frac{-\sigma^2\left|\Delta \vec{K}\right|^2}{2m^2\tau^2}\left(\frac{e^{-\chi''t}}{2\chi'\chi''}\operatorname{Re}\left(\frac{e^{i\chi't}}{\chi\left(\chi^2+\eta_1^2\right)}\right) + \frac{e^{-\eta_1t}}{\left(\left|\chi\right|^2+\eta_1^2\right)^2\eta_1}\right)\right) \label{isf_exp}
\end{equation}
\begin{equation}
	\left<E(0)E(t)\right>=\left<E(0)E(\infty)\right> + \frac{\sigma^4}{4\tau^4m^2}\left(\frac{e^{-\chi''t}}{2\chi'\chi''}\operatorname{Re}\left(\frac{\chi e^{i\chi't}}{\chi^2+\eta_1^2}\right) + \frac{\eta_1e^{-\eta_1 t}}{\left(\left|\chi\right|^2 + \eta_1^2\right)^2} \right)^2
\end{equation}
$$
\left<E(0)E(\infty)\right> = \frac{\sigma^4}{4\tau^4m^2}\left(\frac{1}{2\chi'\chi''}\operatorname{Re}\left(\frac{\chi}{\chi^2+\eta_1^2}\right) - \frac{\eta_1}{\left(\left|\chi\right|^2 + \eta_1^2\right)^2} \right)^2
$$

Where $\chi=:\chi'+\chi''i$.


\bibliographystyle{abbrv}
\bibliography{bibliography}

\end{document}
