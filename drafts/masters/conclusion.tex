
\chapter{Conclusions and Outlook}

A careful analysis of the thermalization properties of the GLE in various background potentials has been performed. The results show that the introduction of a cutoff in the noise spectrum introduces additional variance in the temperature fluctuations of the system. Additionally, the Velocity Verlet integration technique is shown to produce a mean simulation temperature which is $1-2\%$ too high when applied to the GLE in a corrugated potential.

A first-principles derivation of the ISF and energy auto-correlation function of a particle obeying the GLE in a harmonic background potential has been presented. These expressions have been evaluated analytically for the first time in the special case of an exponential memory kernel, $K(t)=\frac{1}{\tau}e^{-\frac{t}{\tau}}$, and used to provide interpretations of the GLE parameters in the small $\tau$ regime. These results serve as a useful tool for extracting GLE parameters from the ISF and energy auto-correlation function of a MD simulation. These parameter extraction techniques have potential to be applied directly to experimental data.

A semi-realistic, classical, 3D molecular dynamics simulation of lithium on copper(111) has been constructed with one to two orders of magnitude more substrate atoms than similar simulations in previous work. The simulation has been fully validated, analyzed and compared with the known properties of lithium on copper(111). In particular, the simulation faithfully reproduces the phonon spectrum of real copper (with a phonon cutoff frequency of $\SI{7.4}{\tera\hertz}$) and therefore fully captures the classical notion of the stochastic noise seen by an adatom. The simulation has been used to show that this classical simulation of lithium on copper(111) fails to reproduce the pre-exponential factor of the hopping rate as well as ratio of fast to slow motion seen in the experimental data.

A corresponding 2D GLE simulation with an exponential memory kernel has been constructed with parameters extracted directly from the MD simulation using the aforementioned parameter extraction technique. This simulation has been shown to produce ISFs which match the MD simulations extremely well. Together with results from \cite{Ward} which show that the experimental lithium on copper data is explained using a GLE with a frequency cutoff of around $\SI{1.25}{\tera\hertz}$, this leads to the conclusion that the effective noise seen by lithium on copper(111) is far more band limited than a classical model would predict. This suggests an additional noise filter of unknown origin is present in the system which effectively band limits the noise spectrum seen by the lithium to around $\SI{1.25}{\tera\hertz}$. Since lithium is a very light atom with a thermal de Broglie wavelength on the order of $\SI{0.5}{\angstrom}$ (smaller than but still comparable to the copper lattice spacing of $\SI{2.54}{\angstrom}$), this filter may be quantum mechanical in nature.

Using a corrugated potential extracted from the full MD simulation, the effect of the parameters $\eta$ and $\tau$ on the ISFs produced in GLE simulations are derived numerically. The simulation results re-affirm the expectation that a frequency cutoff much higher than the typical hopping rate observed in the MD simulation of around $\SI{0.93}{\ips}$(corresponding to very small $\tau$) has very little effect on the long term hopping rate of the particle. The numerical results suggest an asymptotic relationship of $\alpha \propto \sqrt{\frac{\eta}{\tau}}$ for the ISF dephasing rate at larger $\tau$ which is not fully understood and may provide an avenue for further research within transition state theory. 
